% !TeX document-id = {1c0b4298-276a-4038-8e08-c4a1f4846da7}
% !TeX TXS-program:bibliography = txs:///biber
\documentclass[10pt,a4paper]{article}
\usepackage[T1]{fontenc}
\usepackage{graphicx}
\usepackage{mathtools}
\usepackage{amssymb}
\usepackage{amsthm}
\usepackage{thmtools}
\usepackage{xcolor}
\usepackage{nameref}
\usepackage{hyperref}
\usepackage{color}
\usepackage{float}

\usepackage[
backend=biber,
style=authoryear,
natbib=true
]{biblatex}
\addbibresource{../bibliography.bib}

\title{}
\author{\normalsize Exercícios sobre Redução de Dados}
\date{}
\begin{document}
	\maketitle
	
 \paragraph{Exercício 1} Seja $\boldsymbol{X}$ um vetor aleatório $k$-dimensional com distribuição normal $\mathcal{N}(\boldsymbol{\mu},\boldsymbol{\Sigma})$. Mostre que, para qualquer matriz $A$ de tamanho $s \times k$, $A\boldsymbol{X} \sim N(A\boldsymbol{\mu},A\boldsymbol{\Sigma}A')$. \textit{Dica:} use a função característica de uma normal.
 \paragraph{Exercício 2} Considere o experimento dado pela observação de $n\geq 2$ variáveis aleatórias normais independentes, $X_1$,$X_2$,\ldots, $X_n$ com parâmetro de média $\mu$ comum desconhecido e parâmetro de variância $\sigma^2$ \textbf{conhecido}.
 
 \begin{enumerate}
 	\item[a] Qual é o modelo $\mathcal{P}$?
 	\item[b] Mostre que $T(X_1,\ldots, X_n) = \frac{X_1+X_2+\ldots+X_n}{n}$ é uma estatística completa suficiente em  $\mathcal{P}$ .
 	\item[c] Mostre que $\hat{S}^2 = \frac{\sum_{i=1}^n (X_i- T(X_1,\ldots, X_n))^2}{n-1}$ é ancilar em  $\mathcal{P}$. \textit{Dica:} proceda indutivamente, começando de $n=2$.
 	\item[d] Conclua que, para todo valor possível de $\mu$, $T(X_1,\ldots, X_n)$ é independente de $\hat{S}^2$.
 \end{enumerate}
 \paragraph{Exercício 3} Considere o experimento dado pela observação de $n$ variáveis aleatórias independentes com distribuição $U[0,\theta]$, com $\theta > 0$ desconhecido.
 \begin{itemize}
 	\item[a] Qual é o modelo $\mathcal{P}$?
 	\item[b] Mostre que $X_{(n)} \coloneqq \max_{1\leq i \leq n} X_i$ é estatística suficiente para $\mathcal{P}$.
 	\item[c] Mostre que a lei de probabilidade induzida por $X_{(n)}$ admite densidade com respeito à medida de Lebesgue $\lambda$ em $\mathbb{R}_+$ dada por:
 	
 	$$g_\theta(z) \coloneqq \frac{n z^{n-1}}{\theta^n}\mathbf{1}_{[0,\theta]}(z)$$
 	
 	\item[d] Mostre que, para qualquer transformação $f: \mathbb{R} \mapsto \mathbb{R}$ tal que $\mathbb{E}_\theta[f(X_{(n)})] = 0$ para todo $\theta > 0$, devemos ter que:
 	
 	$$\int t^{n-1} f^+(t) \mathbf{1}_{[0,\theta]}\lambda(dt) = \int t^{n-1} f^-(t) \mathbf{1}_{[0,\theta]}\lambda(dt)\, .$$
 	
 	Pela extensão do lema do $\pi$-sistema vista na Lista 1 do curso de verão, esse fato implicará que:
 	
 	$$\int t^{n-1} f^+(t) \mathbf{1}_{B}(t)\lambda(dt) = \int t^{n-1} f^-(t) \mathbf{1}_{B}(t)\lambda(dt)\, ,\quad \forall B \in \mathcal{B}(\mathbb{R}_+)\, .$$
 	
 	Usando o fato acima, mostre que $f(t) = 0$ em $\lambda$-quase-todo ponto. Conclua que $X_{(n)}$ é estatística completa suficiente..
 	
 	\item[e] Encontre um estimador não viciado de variância uniformemente mínima para $\theta$. \textit{Dica:} $2 X_1$ é estimador não viciado de $\theta$.
 	
 	
 \end{itemize}
 
 \paragraph{Exercício 4} Considere uma família exponencial $\{P_\eta : \eta \in \Xi\}$ com respeito a uma medida $\mu$ em que as densidades são da forma:
 
 $$p_\theta(x) = \exp(\eta T(x) - B(\eta)) h(x)\, ,$$
 com $\xi \in \mathbb{R}$ e $T(x) \in \mathbb{R}$. O espaço $\Xi$ é o conjunto de pontos em que: 
 
 $$\int \exp(\eta T(x)) h(x) \mu(dx) < \infty$$
 
 No que segue, vamos supor que esse conjunto é \textbf{aberto}.
 
 \begin{itemize}
 	\item[a]Usando que $p_\eta$ é densidade,  encontre uma expressão fechada para $B(\eta)$.
 	\item[b] Mostre que, para todo $\eta \in \Xi$, a função geradora de momentos de $T(X)$ existe numa vizinhança de zero, e é dada, nesta vizinhança, por:
 	
 	$$M_{T(X)|\eta}(u) =  \exp(B(\eta+u) - B(\eta))$$
 	
 	\item[c] Mostre que $\mathbb{E}_\eta[T(X)] = B'(\eta)$ para todo $\eta \in \Xi$.
 	\item[d] Mostre que a variância de $T(X)$, visto enquanto estimador de  $B'(\eta)$, atinge o limite inferior de Crámer-Rao.
 \end{itemize}
\end{document}