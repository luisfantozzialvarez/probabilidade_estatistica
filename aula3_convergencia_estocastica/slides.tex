% !TeX document-id = {0be8c18c-9430-4e9a-bdd9-12beadebfebc}
% !TeX TXS-program:bibliography = txs:///biber
\documentclass[11pt]{beamer}
\uselanguage{portuguese}
\languagepath{portuguese}
\deftranslation[to=portuguese]{Theorem}{Teorema}
\deftranslation[to=portuguese]{Definition}{Definição}
\deftranslation[to=portuguese]{theorem}{teorema}
\deftranslation[to=portuguese]{Example}{Exemplo}
\deftranslation[to=portuguese]{example}{exemplo}
\deftranslation[to=portuguese]{Lemma}{Lema}
\deftranslation[to=portuguese]{lemma}{Lema}
\deftranslation[to=portuguese]{Corollary}{Corolário}
\deftranslation[to=portuguese]{corollary}{corolário}
%\deftranslation[to=portuguese]{and}{e}


\usepackage[brazilian]{babel}
\usepackage[utf8]{inputenc}
\usepackage[T1]{fontenc}
\usepackage{lmodern}
\usepackage{amsmath}
\usepackage{amssymb}
\usepackage{mathtools}
\usepackage{color}
\usepackage{pgfplots}
\usepackage{tikz}
\usepackage{subcaption}
%\usepackage{appendixnumberbeamer}

\newenvironment{transitionframe}{
	\setbeamercolor{background canvas}{bg=yellow}
	\begin{frame}}{
	\end{frame}
}
\usetheme{default}
\usefonttheme{structuresmallcapsserif}

%% I use a beige off white for my background
\definecolor{MyBackground}{RGB}{255,253,218}
\useinnertheme[shadow]{rounded}
\setbeamercolor{block title}{bg=MyBackground}
\setbeamercolor{block body}{bg=MyBackground}
\setbeamercolor{example title}{bg=MyBackground}
\setbeamercolor{example body}{bg=MyBackground}


\newcommand{\blue}[1]{\textcolor{blue}{#1}}
\newcommand{\red}[1]{\textcolor{red}{#1}}
\newcommand{\purple}[1]{\textcolor{purple}{#1}}
\newcommand{\gray}[1]{\textcolor{gray}{#1}}
\setbeamertemplate{navigation symbols}{}
%\setbeamertemplate{page number in head/foot}[appendixframenumber]

%\usepackage{graphics}
\usepackage{graphicx}

\definecolor{blue_emph}{RGB}{0,114,178}
\definecolor{red}{RGB}{213,94,0}
\definecolor{yellow}{RGB}{240,228,66}
\definecolor{green}{RGB}{0,158,115}
\definecolor{purple}{RGB}{204,121,167}
\definecolor{orange}{RGB}{230,159,0}
\definecolor{lightblue}{RGB}{86,180,233}

%\setbeamercolor{frametitle}{fg=blue}
%\setbeamercolor{title}{fg=blue}
\setbeamertemplate{footline}[frame number]
\setbeamertemplate{navigation symbols}{} 
\setbeamertemplate{itemize items}{-}
%\setbeamercolor{itemize item}{fg=blue}
%\setbeamercolor{itemize subitem}{fg=blue}
\setbeamertemplate{enumerate items}[default]
%\setbeamercolor{enumerate subitem}{fg=blue}
\setbeamercolor{button}{bg=MyBackground,fg=blue}
\usefonttheme{structuresmallcapsserif}

%\setbeamercolor{section in toc}{fg=blue}
%\setbeamercolor{subsection in toc}{fg=red}
\setbeamersize{text margin left=1em,text margin right=1em} 


\usepackage{appendixnumberbeamer}

\usepackage[
backend=biber,
style=authoryear,
natbib=true
]{biblatex}
\addbibresource{../bibliography.bib}

\newenvironment{wideitemize}{\itemize\addtolength{\itemsep}{10pt}}{\enditemize}
\newenvironment{wideenumerate}{\enumerate\addtolength{\itemsep}{10pt}}{\endenumerate}
\newenvironment{halfwideitemize}{\itemize\addtolength{\itemsep}{0.5em}}{\enditemize}
\newenvironment{halfwideenumerate}{\enumerate\addtolength{\itemsep}{0.5em}}{\endenumerate}

\def\signed #1{{\leavevmode\unskip\nobreak\hfil\penalty50\hskip2em
		\hbox{}\nobreak\hfil(#1)%
		\parfillskip=0pt \finalhyphendemerits=0 \endgraf}}

\newsavebox\mybox
\newenvironment{aquote}[1]
{\savebox\mybox{#1}\begin{quote}}
	{\signed{\usebox\mybox}\end{quote}}

\newtheorem{proposition}{Proposição}



\author{Luis A. F. Alvarez}
\title{Probabilidade e Estatística}
\subtitle{Aula 3 -- Convergência estocástica}
%\logo{}
%\institute{}
\date{\today}
%\subject{}
%\setbeamercovered{transparent}

\begin{document}
	
	\maketitle
	
	\begin{frame}{Vetores aleatórios}
		\begin{itemize}
			\item Seja $(\Omega,\Sigma,\mathbb{P})$ um espaço de probabilidade.
			\item Um vetor aleatório $\boldsymbol{Y}: \Omega \mapsto \mathbb{R}^k$ é uma função tal que cada coordenada $\boldsymbol{Y}_l : \Omega \mapsto \mathbb{R}$, $l=1,\ldots, k$, é uma variável aleatória real.
			\item Um vetor aleatório induz uma distribuição de probabilidade sobre $(\mathbb{R}^k,\mathcal{B}(\mathbb{R}^k))$, dada por $\mathbb{P}_{\boldsymbol{Y}}[B] = \mathbb{P}[\boldsymbol{Y}^{-1}(A)]$, $A \in \mathcal{B}(\mathbb{R}^k)$.
			\item Pelo lema do $\pi$-sistema, essa distribuição de probabilidade é carectarizada pela função de distribuição $F_{\boldsymbol{Y}}: \mathbb{R}^k \mapsto [0,1]$, dada por:
			$$F_{\boldsymbol{Y}}(\boldsymbol{c}) \coloneqq \mathbb{P}_{\boldsymbol{Y}}\left[\prod_{l=1}^k (-\infty,\boldsymbol{c}_k]\right]\, , \quad \boldsymbol{c} \in \mathbb{R}^k\, .$$
			
		\end{itemize}

	\end{frame}
	\begin{frame}{Convergência quase-certa}
		\begin{itemize}
			\item  Seja $(\Omega,\Sigma,\mathbb{P})$ um espaço de probabilidade, e $\boldsymbol{Y}_1,\boldsymbol{Y}_2,\ldots$ uma sequência de vetores aleatórios.
			\item Dizemos que $\boldsymbol{Y}_n$ converge quase-certamente para um vetor aleatório $\boldsymbol{Y}$, denotado por $\boldsymbol{Y}_n \overset{\text{q.c.}}{\to} \boldsymbol{Y}$, se:
			
			$$\mathbb{P}[\{\omega: \boldsymbol{Y}_n(\omega) \nrightarrow \boldsymbol{Y}(\omega)\}] = 0\, .$$
			\item Sequência de funções $\boldsymbol{Y}_n$ convergem (ponto a ponto), a não ser num conjunto de pontos de probabilidade zero.
			
			\begin{lemma}
				$\boldsymbol{Y}_n \overset{\text{q.c.}}{\to} \boldsymbol{Y}$ se, e somente se, para todo $\epsilon > 0$:
				
			$$\mathbb{P}\left[\limsup_n \{\omega: \lVert\boldsymbol{Y}_n(\omega)-\boldsymbol{Y}(\omega)\rVert > \epsilon\}\right]=0\, .$$
			\end{lemma} 
		\end{itemize}

	\end{frame}
	
	\begin{frame}{Convergência em probabilidade}
		\begin{itemize}
				\item Dizemos que $\boldsymbol{Y}_n$ converge em probabilidade para um vetor aleatório $\boldsymbol{Y}$, denotado por $\boldsymbol{Y}_n \overset{p}{\to} \boldsymbol{Y}$, se, para todo $\epsilon > 0$:
					$$\lim_{n \to \infty} \mathbb{P}\left[\{\omega: \lVert\boldsymbol{Y}_n(\omega)-\boldsymbol{Y}(\omega)\rVert > \epsilon\}\right]= 0\, .$$
				
		\end{itemize}
		\begin{lemma}
			Se $\boldsymbol{Y}_n \overset{\text{q.c.}}{\to} \boldsymbol{Y}$, então $\boldsymbol{Y}_n \overset{p}{\to} \boldsymbol{Y}$.
		\end{lemma}
	\end{frame}
	
	\begin{frame}{Convergência em distribuição}
			\begin{itemize}
			\item Dizemos que $\boldsymbol{Y}_n$ converge em probabilidade para um vetor aleatório $\boldsymbol{Y}$, denotado por $\boldsymbol{Y}_n \overset{p}{\to} \boldsymbol{Y}$, se, para todo $\epsilon > 0$:
			$$\lim_{n \to \infty} \mathbb{P}\left[\{\omega: \lVert\boldsymbol{Y}_n(\omega)-\boldsymbol{Y}(\omega)\rVert > \epsilon\}\right]= 0\, .$$
			
		\end{itemize}
		\begin{lemma}
			Se $\boldsymbol{Y}_n \overset{\text{q.c.}}{\to} \boldsymbol{Y}$, então $\boldsymbol{Y}_n \overset{p}{\to} \boldsymbol{Y}$.
		\end{lemma}
	\end{frame}
	
	
\end{document}
