% !TeX document-id = {0be8c18c-9430-4e9a-bdd9-12beadebfebc}
% !TeX TXS-program:bibliography = txs:///biber
\documentclass[11pt]{beamer}
\uselanguage{portuguese}
\languagepath{portuguese}
\deftranslation[to=portuguese]{Theorem}{Teorema}
\deftranslation[to=portuguese]{Definition}{Definição}
\deftranslation[to=portuguese]{theorem}{teorema}
\deftranslation[to=portuguese]{Example}{Exemplo}
\deftranslation[to=portuguese]{example}{exemplo}
\deftranslation[to=portuguese]{Lemma}{Lema}
\deftranslation[to=portuguese]{lemma}{Lema}
\deftranslation[to=portuguese]{Corollary}{Corolário}
\deftranslation[to=portuguese]{corollary}{corolário}
%\deftranslation[to=portuguese]{and}{e}


\usepackage[brazilian]{babel}
\usepackage[utf8]{inputenc}
\usepackage[T1]{fontenc}
\usepackage{lmodern}
\usepackage{amsmath}
\usepackage{amssymb}
\usepackage{mathtools}
\usepackage{color}
\usepackage{pgfplots}
\usepackage{tikz}
\usepackage{subcaption}
%\usepackage{appendixnumberbeamer}

\newenvironment{transitionframe}{
	\setbeamercolor{background canvas}{bg=yellow}
	\begin{frame}}{
	\end{frame}
}
\usetheme{default}
\usefonttheme{structuresmallcapsserif}

%% I use a beige off white for my background
\definecolor{MyBackground}{RGB}{255,253,218}
\useinnertheme[shadow]{rounded}
\setbeamercolor{block title}{bg=MyBackground}
\setbeamercolor{block body}{bg=MyBackground}
\setbeamercolor{example title}{bg=MyBackground}
\setbeamercolor{example body}{bg=MyBackground}


\newcommand{\blue}[1]{\textcolor{blue}{#1}}
\newcommand{\red}[1]{\textcolor{red}{#1}}
\newcommand{\purple}[1]{\textcolor{purple}{#1}}
\newcommand{\gray}[1]{\textcolor{gray}{#1}}
\setbeamertemplate{navigation symbols}{}
%\setbeamertemplate{page number in head/foot}[appendixframenumber]

%\usepackage{graphics}
\usepackage{graphicx}

\definecolor{blue_emph}{RGB}{0,114,178}
\definecolor{red}{RGB}{213,94,0}
\definecolor{yellow}{RGB}{240,228,66}
\definecolor{green}{RGB}{0,158,115}
\definecolor{purple}{RGB}{204,121,167}
\definecolor{orange}{RGB}{230,159,0}
\definecolor{lightblue}{RGB}{86,180,233}

%\setbeamercolor{frametitle}{fg=blue}
%\setbeamercolor{title}{fg=blue}
\setbeamertemplate{footline}[frame number]
\setbeamertemplate{navigation symbols}{} 
\setbeamertemplate{itemize items}{-}
%\setbeamercolor{itemize item}{fg=blue}
%\setbeamercolor{itemize subitem}{fg=blue}
\setbeamertemplate{enumerate items}[default]
%\setbeamercolor{enumerate subitem}{fg=blue}
\setbeamercolor{button}{bg=MyBackground,fg=blue}
\usefonttheme{structuresmallcapsserif}

%\setbeamercolor{section in toc}{fg=blue}
%\setbeamercolor{subsection in toc}{fg=red}
\setbeamersize{text margin left=1em,text margin right=1em} 


\usepackage{appendixnumberbeamer}

\usepackage[
backend=biber,
style=authoryear,
natbib=true
]{biblatex}
\addbibresource{../bibliography.bib}

\newenvironment{wideitemize}{\itemize\addtolength{\itemsep}{10pt}}{\enditemize}
\newenvironment{wideenumerate}{\enumerate\addtolength{\itemsep}{10pt}}{\endenumerate}
\newenvironment{halfwideitemize}{\itemize\addtolength{\itemsep}{0.5em}}{\enditemize}
\newenvironment{halfwideenumerate}{\enumerate\addtolength{\itemsep}{0.5em}}{\endenumerate}

\def\signed #1{{\leavevmode\unskip\nobreak\hfil\penalty50\hskip2em
		\hbox{}\nobreak\hfil(#1)%
		\parfillskip=0pt \finalhyphendemerits=0 \endgraf}}

\newsavebox\mybox
\newenvironment{aquote}[1]
{\savebox\mybox{#1}\begin{quote}}
	{\signed{\usebox\mybox}\end{quote}}

\newtheorem{proposition}{Proposição}



\author{Luis A. F. Alvarez}
\title{Probabilidade e Estatística}
\subtitle{Aula 2 -- Integração e Expectativa}
%\logo{}
%\institute{}
\date{\today}
%\subject{}
%\setbeamercovered{transparent}

\begin{document}
	
	\maketitle
\begin{frame}{Integral de Lebesgue}
	\begin{itemize}
		\item Nesta  aula, partiremos de um espaço de medida $(\Omega, \Sigma,\mu)$ e definiremos, para uma função mensurável $f: \Omega \mapsto \mathbb{R}$, a integral de Lebesgue com respeito a $\mu$.
		\begin{itemize}
			\item Como veremos, essa noção de integral \emph{estende} a noção de integral de Riemann para uma classe mais ampla de funções e espaços subjacentes.
			\item Esse conceito de integral será fundamental para a definição formal de esperança condicional.
		\end{itemize}
	\end{itemize}
\end{frame}

\begin{frame}{Funções simples}
	\begin{itemize}
	\item Seja $(\Omega, \Sigma, \mu)$ um espaço de medida.
	\item Uma função $f:\Omega \mapsto \mathbb{R} \cup \{-\infty,\infty\}$ é dita simples se existem $k \in \mathbb{N}$ e $E_1, E_2,\ldots, E_k \in \Sigma$, $a_1,\ldots, a_k \in \mathbb{R}\cup \{-\infty,\infty\}$ tais que:
		$$f(\omega) = \sum_{i=1}^k a_i \mathbf{1}_{E_i}(\omega)\, , \quad \omega \in \Omega$$
		onde $\mathbf{1}_A(\omega) = \begin{cases}
			1 & \text{se } \omega \in A \\
			0 & \text{se } \omega \notin A
		\end{cases}$ é a função indicadora do conjunto $A$.
		\item \textbf{Convenção:} se $\mathbf{1}_{E_i}(\omega) =0$ e $a_i = \pm \infty$, então $\mathbf{1}_{E_i}(\omega) a_i = 0$
		\item Fácil ver que $f$ é mensurável.
		\begin{itemize}
			\item Também fácil ver que, sem perda de generalidade, podemos tomar os $E_i$ como disjuntos.
		\end{itemize}
		
	\end{itemize}
	\end{frame}
	
	\begin{frame}{Integral de Lebesgue de funções simples não negativas}
		\begin{itemize}
				\item Seja $f$ uma função simples \textbf{não negativa}. A integral de Lebesgue com respeito a $\mu$, denotada por $\mu(f)$ ou $\int f(\omega) \mu(d\omega)$ ou $\int f d \mu$, é definida por:
			$$\int f(\omega) \mu(d\omega)\coloneqq \sum_{i=1}^k a_i \mu(E_i) \, .$$
			\item Fácil de ver que integral está bem-definida (para duas expressões distintas da função simples em termos de conjuntos finitos, expressão dará a mesma coisa).
			\item Integral será um elemento de $[0,\infty]$
		\end{itemize}
	
	\end{frame}
	
		\begin{frame}{Integral de Lebesgue de funções mensuráveis não negativas}
		\begin{itemize}
			\item Seja $f: \Omega \mapsto [0,\infty]$ uma função mensurável não negativa. Definimos a integral de Lebesgue  como:
			
			$$\mu(f) \coloneqq \sup (\mu(g): g\text{ simples e não negativa}, g \leq f)\, .$$
			
		\begin{lemma}
			Seja $f \geq 0$ mensurável. 
			\begin{itemize}
				\item Se $\mu(f)  < \infty$, então, $\mu(\{\omega: f(\omega) = \infty\}) = 0$.
				\item Se $\mu(f) = 0$, então $\mu(\{\omega: f(\omega) > 0\}) = 0$.
				\item Seja $g \geq 0$ mensurável. Se $\mu(\{\omega: g(\omega)\neq f(\omega)\})=0$, então $\mu(f) = \mu(g)$.
			\end{itemize}
		\end{lemma}
		\vspace{0.1em}
		\item 	\textbf{Obs:} Quando uma afirmação é válida a não ser em um conjunto de pontos $\omega$ de medida zero, dizemos que ela vale em $\mu$-quase todo ponto ($\mu$-q.t.p.).
		\begin{itemize}
			\item Se  $\mu$ é medida de probabilidade, equivalente ao qualificador ``quase certamente'' visto em aula anterior.
		\end{itemize}
		\end{itemize}
	
	\end{frame}
	
	\begin{frame}{Teorema da Covergência Monótona}
		\begin{itemize}
	\item Seja $(f_n)_n$ uma sequência de funções mensuráveis. Dizemos que $f_n$ converge a uma função $f$ em $\mu$-quase todo ponto se $\mu(\{\omega: f_n(\omega) \nrightarrow f(\omega)\})) = 0$.
	\begin{itemize}
		\item Medida do evento em que não há convergência é zero.
	\end{itemize}
	\item 	Como $f$  é mensurável (por quê?),  se $f_n \geq 0$ para todo $n$, podemos nos perguntar se:
	
	$$\int f_n d \mu \to \int f d \mu\, ,$$
	i.e. podemos passar o limite por ``debaixo'' da integral?
	\item Resposta é verdadeira se a convergência for \textbf{monótona}  em $\mu$-quase todo ponto, i.e. $\mu(\{\omega: f_n(\omega) \uparrow f(\omega)\}^\complement) = 0$.
	\begin{theorem}
		Seja $(f_n)_n$ uma sequência de funções não negativas mensuráveis tais que $f_n \uparrow f$ em $\mu$-quase todo ponto. Então.
		
		$$\mu(f_n) \uparrow \mu(f)\leq \infty$$
	\end{theorem}
	\end{itemize}
	\end{frame}
	
	\begin{frame}{Construindo aproximação monotônicas}
\begin{itemize}
	\item Em alguns contextos, é interessante construir funções monotônicas que aproximam uma dada função não negativa $f$. 
	\item Uma construção bastante comum é dada por $f_n = s_n \circ f$, onde $s_n$ são funções escada da forma:
	
	$$s_n(y) = \begin{cases}
		0 & \text{se } y = 0\\
		\frac{(j-1) n}{2^n}\,, &  \text{se } y \in \left(	\frac{(j-1) n}{2^n} , 	\frac{j n}{2^n}\right], j \in \{1,\ldots, 2^n\} \\
		n,, & \text{se } y > n
	\end{cases}$$
	\item Sequência é tal que $f_n \uparrow f$.
\end{itemize}

	\end{frame}
	
	\begin{frame}{Integral de Lebesgue no caso geral}
	\begin{itemize}
		\item 		Seja $f:\Omega \mapsto \mathbb{R}\cup\{\infty,-\infty\}$ uma função mensurável. Defina as partes positiva e negativa de $f$ como:
		
		$$f^+ = \max\{f,0\} \,, \quad f^-  = - \min\{f,0\}\, ,$$
		

		\item Dizemos que $f$ é \textbf{integrável} se $\mu(|f|) = \mu(f^+) + \mu(f^-) < \infty$. Nesse caso, a integral de $f$ é definida como:
		
		$$\mu(f) = \mu(f^+)-\mu(f^-)$$
		
		\item Vamos denotar por $L^1(\Omega, \Sigma,\mu)$ o espaço de funções Lebesgue-integráveis.
		
	\end{itemize}
		
	\end{frame}
	\begin{frame}{Integral de Lebesgue: propriedades}
	\begin{proposition}
		Sejam $f,g \in L^1(\Omega,\Sigma, \mu)$ e $\lambda \in \mathbb{R}$. Temos que:
		\begin{itemize}
			\item[1] (Monotonicidade) $f \leq g \implies \mu(f)\leq \mu(g)$
			\item[2] (Linearidade) $f + \lambda g \in  L^1(\Omega,\Sigma, \mu)$ e $\mu(f+\lambda g) = \mu(f)+\lambda\mu(g)$.
		\end{itemize}
	\end{proposition}
	\begin{proposition}
		Seja $(f_n)_n$ uma sequência de funções em $ L^1(\Omega,\Sigma, \mu)$, tais que $f_n \to f$ em $\mu$-q.t.p. Se existe $g\geq $ mensurável  tal que $\mu(g) < \infty$ e:
		
		$$|f_n(\omega)|\leq g(\omega)\,  \forall n \in \mathbb{N}, \omega \in \Omega\, ,$$
		então  $f \in L^1(\Omega,\Sigma, \mu)$ e: 
		
		$$\mu(|f_n-f|)\to 0 \,,$$ $$ \mu(f_n) \to \mu(f)$$
	\end{proposition}
	\end{frame}
	
	\begin{frame}{Integral de Lebesgue e integral de Riemann}
		\begin{itemize}
			\item 	 Considere o espaço $([0,1], \mathcal{B}[0,1], \lambda)$, com $\lambda$ a medida de Lebesgue.
			\item Nesse espaço, podemos tanto calcular a integral de Lebesgue $\int f(x)\lambda(dx)$ como a integral de Riemann:
			$$\int_{0} ^1 f(x) dx$$
			\item Qual a relação entre as duas integrais?
			\begin{proposition}
				Seja $f\geq$ uma função real com domínio em $[0,1]$ Riemann integrável. Então $f$ é mensurável e $\lambda(f) = \int_0^1 f(x) dx$.
			\end{proposition}
			\item A recíproca do resultado acima nem sempre verdadeira. Por exemplo, a função
			$$\mathbf{1}_{[0,1]\setminus  \mathbb{Q}}\, ,$$
			não é Riemann-integrável, embora $\lambda(\mathbf{1}_{[0,1]\setminus  \mathbb{Q}}) = 0$.
		\end{itemize}

	\end{frame}
	
	\begin{frame}{Densidade com respeito a uma medida}

	\begin{itemize}
		\item 	Seja $(\Omega, \Sigma, \mu)$ um espaço de medida, e $f\geq0$ uma função mensurável.
		\item O conjunto de integrais:
		
		$$\int_{A} f d \mu \coloneqq \mu(f \mathbf{1}_{A})\, , A \in \Sigma$$
		
		define uma medida sobre $(\Omega, \Sigma)$ (verifique).
		\item Reciprocamente, se $\Phi$ é uma medida sobre $(\Omega,\Sigma,\mu)$, dizemos que $\Phi$ admite uma densidade com respeito a $\mu$ se existe $g \geq 0$ mensurável tal que, para todo $A \in \Sigma$:
		$$\Phi(A) = \int_{A} g d \mu$$
		\item Condição necessária e suficiente para existência de densidade é dada pelo teorema de Radon-Nikodyn.
		\begin{theorem}
			Seja $(\Omega,\Sigma)$ um espaço mensurável, e $\mu$ e $\Phi$ duas medidas {\color{red}$\sigma$-finitas}. $\Phi$ admite uma densidade com respeito a $\mu$ se, e somente se, para todo $A \in \Sigma$, $\mu(A) \implies \Phi(A)$.
		\end{theorem}
	\end{itemize}

	\end{frame}
	\begin{frame}{Densidade e uma fórmula padrão}
		\begin{lemma}
			Seja $(\Omega,\Sigma)$ um espaço mensurável, e $\mu$ e $\Phi$ duas medidas.  Se $\Phi$ admite densidade $g$ com respeito a a $\mu$, então para qualquer $h \in L^1(\Omega, \Sigma, \Phi)$, temos que:
			$$\Phi(f) = \int h(\omega) g(\omega) \mu(d\omega)$$
		\end{lemma}
		\begin{itemize}
			\item Fórmula acima nos permite calcular esperança diretamente da integral com respeito a $\mu$.
			\item \textbf{Demonstração:} primeiro verificamos  a expressão para funções simples, depois para funções não negativas usando uma aproximação por funções-escada, e por fim estendemos para funções gerais.
		\end{itemize}
	\end{frame}
	\begin{frame}{Esperança}
	\begin{itemize}
		\item Seja $(\Omega, \Sigma\,\mathbb{P})$ espaço de probabilidade.
		\item Neste caso, damos à integral de Lebesgue o nome de expectativa ou esperança, denotando-a, para $X \in L^1(\Omega,\Sigma, \mathbb{P})$ por:
		$$\mathbb{E}[X] \coloneqq \mathbb{P}(X)$$.
		\item Como uma medida de probabilidade é finita, um corolário imediato do teorema da convergência dominada é:
		\begin{corollary}[Teorema da convergência limitada]
			Seja $(X_n)_n \in L^1(\Omega, \Sigma,\mathbb{P})^{\mathbb{N}}$ tais que $X_n \to X$ quase certamente. Se existe $K>0$ tal que:
			
			$$\mathbb{P}[\{\omega:|X_n(\omega)|\leq K,\,  \forall n\}] = 1\, ,$$
			então $X \in L^1(\Omega, \Sigma,\mathbb{P})$ e:
			$\mathbb{E}[X_n] \to \mathbb{E}[X]$
		\end{corollary}
	\end{itemize}
	\end{frame}
	
	\begin{frame}{Esperança: desigualdades fundamentais}
		\begin{itemize}
			\item No que segue, considere um espaço de probabilidade $(\Omega,\Sigma, \mathbb{P})$.
		\end{itemize}
		\begin{lemma}{Desigualdade de Markov}
			Seja $Z$ uma variável aleatória, e $g: \mathbb{R} \mapsto [0,\infty]$ mensurável e não decrescente.  Então, para todo $c \in \mathbb{R}$.
			
			$$\mathbb{P}[Z\geq c] g(c) \leq \mathbb{E}[g(Z)]$$
		\end{lemma}
		
			\begin{lemma}{Desigualdade de Jensen}
		Seja $X$ uma variável aleatória, e $c: C \mapsto \mathbb{R}$ uma função convexa onde $C \subseteq \mathbb{R}$ é um conjunto aberto. Suponha que:
		
		$$\mathbb{E}[|X|]< \infty, \quad \mathbb{P}[X \in C]  = 1, \quad \mathbb{E}[|c(X)|]<\infty\, ,$$
		então $\mathbb{E}[X] \in C$ e:
		$$c(\mathbb{E}[X])\leq \mathbb{E}[c(X)]\, .$$
		\end{lemma}
	\end{frame}
	
	\begin{frame}{A norma $L^p$}
	\begin{itemize}
		\item Fixe $p \in [1,\infty]$. Para uma variável aleatória $X$, nós definimos:
		$$\lVert X\rVert_p = (\mathbb{E}[|X|^p])^{1/p}$$
		\item Denotamos por $L^p(\Omega, \Sigma, \mathbb{P})$ o espaço de variáveis aleatórias $X \in m(\Sigma)$ tais que $\lVert X\rVert_p<\infty$.
		\item É possível mostrar que esse espaço é linear normado, com $\lVert \cdot \rVert_p$ definindo uma (semi)norma em $L^p(\Omega, \Sigma, \mathbb{P})$ .
		\begin{itemize}
			\item ``Semi'' vem do fato de que $\lVert X \rVert_p = 0 \implies \mathbb{P}[X=0]=1$, de modo que há múltiplas variáveis aleatórias com norma zero, embora todas-quase certamente iguais a zero. 
			\begin{itemize}
				\item Essa multiplicidade não é problemática.
			\end{itemize}
		\end{itemize}
	
	\end{itemize}
	\end{frame}
	\begin{frame}{Normas $L_p$: propriedades úteis}
\begin{itemize}
	\item Abaixo, elencamos algumas propriedades úteis das normas $L_p$.
	\begin{lemma}
		\begin{enumerate}
			\item (Monotonicidade) Sejam $1 \leq p \leq q$. Se $X \in L^q(\Omega, \Sigma, \mathbb{P})$ , então $X \in L^p(\Omega, \Sigma, \mathbb{P})$ e $\lVert X \rVert_p \leq \lVert X \rVert_q$.
				\item (Cauchy-Schwarz) sejam $X,Y \in  L^2(\Omega, \Sigma, \mathbb{P})$. Então $X \cdot Y \in L^1(\Omega, \Sigma, \mathbb{P})$ e :
			$$|\mathbb{E}[XY]| \leq \mathbb{E}[|XY|] \leq \lVert X \rVert_2 \lVert Y \rVert_2\, .$$
			\item (H\"{o}lder) Seja $1 \leq p \leq q \leq \infty$, $\frac{1}{p}+\frac{1}{q} =1$. Sejam $X \in  L^p(\Omega, \Sigma, \mathbb{P})$ e  $Y \in  L^q(\Omega, \Sigma, \mathbb{P})$. Então $X \cdot Y \in L^1(\Omega, \Sigma, \mathbb{P})$  e: 
			$$|\mathbb{E}[XY] | \leq \mathbb{E}[|XY|]  \leq \lVert X \rVert_p   \lVert Y \rVert_q\, .   $$
			\item (Minkowski) Seja $p \geq 1$, e $X,Y \in  L^p(\Omega, \Sigma, \mathbb{P})$, então:
			$$\lVert X+Y\rVert_p \leq \lVert X\rVert_p+\lVert Y\rVert_p $$
		\end{enumerate}
	\end{lemma}
	\begin{itemize}
		\item Cauchy-Schwarz é caso particular de H\"{o}lder.
		\item Minkowski garante desigualdade triangular (e que $\lVert \cdot \rVert_p$ é norma)
	\end{itemize}
\end{itemize}
	\end{frame}
	
	\begin{frame}{Esperança condicional}
	\begin{itemize}
		\item Seja $(\Omega, \Sigma,\mathbb{P})$ um espaço de probabilidade, $Y$ uma variável aleatória, e $\mathcal{G}\subseteq \Sigma$ uma sub-$\sigma$-álgebra de $\Sigma$.
		\item Tome $Y \in L^1(\Omega, \Sigma,\mathbb{P})$.
		\item Definimos a esperança condicional de $Y$ com respeito a $\mathcal{G}$, denotada por $\mathbb{E}[Y|\mathcal{G}]$ como a {\color{blue}variável aleatória} $\mathbb{E}[Y|\mathcal{G}] \in L^1(\Omega, {\color{red}\mathcal{G}},\mathbb{P})$  que satisfaz, para todo $A \in {\color{red}\mathcal{G}}$:
		
		$$\mathbb{E}[\mathbb{E}[Y|\mathcal{G}]\mathbf{1}_A] = \mathbb{E}[Y|\mathbf{1}_A]$$
		\item Fácil mostrar que esperança condicional está unicamente definida, a não por eventos de probabilidade zero.
		\begin{itemize}
			\item Uma variável aleatória que satisfaz as condições acima é conhecida como \emph{versão} da esperança condicional.
			\item Sejam $Z_1$ e $Z_2$ duas versões de $\mathbb{E}[Y|\mathcal{G}]$, então $\{\omega: Z_1(\omega)>Z_2(\omega)\} \in \mathcal{G}$ e pela definição da esperança condicional $Z_1\geq Z_2$ q.c.
		\end{itemize}
	\end{itemize}
	\end{frame}

	
	%\begin{frame}[allowframebreaks]{Bibliografia}
	%\printbibliography
	%\end{frame}
		\begin{frame}{Existência da esperança condicional}
		\begin{itemize}
			\item No slide anterior, definimos a esperança condicional e verificamos que, caso exista, ela é única.
			\begin{itemize}
				\item No entanto, cabe a pergunta: será que existe uma variável aleatória que satisfaz as condições requeridas?
						\item Resposta é {\color{red}afirmativa}, e dada por um teorema devido a Komogorov.
			\end{itemize}
			\item Além disso, note que, pelo resultado visto em aula anterior, quando $\mathcal{G} = \sigma(X_1,\ldots, X_n)$, a esperança condicional $\mathbb{E}[Y|\sigma(X_1,\ldots, X_n)] = f(X_1,\ldots,X_n)$ para alguma $f$ $\mathcal{B}(\mathbb{R}^n)$-mensurável.
			\begin{itemize}
				\item Essa é $f$ é conhecida como \textbf{função de expectativa condicional}.
				\item Nesses casos, costumeiro usar a notação $\mathbb{E}[Y|X_1,\ldots, X_n]$ para $\mathbb{E}[Y|\sigma(X_1,\ldots, X_n)]  $
			\end{itemize}
			\item \textbf{Interpretação da esperança condicional:} $\mathcal{G}$ é o conjunto informacional do agente (após sorteio de $\omega \in \Omega$ pela natureza, agente observa se $\omega \in E$ é verdade ou não, para todo $E \in \mathcal{G}$).
			\begin{itemize}
				\item $\omega \mapsto \mathbb{E}[Y|\mathcal{G}]$ é a melhor previsão do agente sobre $Y$ após o sorteio, em termos de minimização do erro quadrático médio, dado o conhecimento de $\mathcal{G}$ (exercício da lista).
			\end{itemize}
		\end{itemize}
	\end{frame}
	\begin{frame}{Esperança condicional: propriedades básicas}
	
			No que segue, tome $X \in \mathcal{L}^1(\Omega,\Sigma,\mathbb{P})$, e $\mathcal{F}$ e $\mathcal{G}$ sub-$\sigma$-álgebras de $\Sigma$.
	\begin{lemma}
			\begin{enumerate}
			\item \textbf{Preservação da Esperança:}  
			Se \( Y \) é qualquer versão de \( \mathbb{E}(X \mid \mathcal{G}) \), então \( \mathbb{E}(Y) = \mathbb{E}(X) \).
			
			\item \textbf{Mensurabilidade:}  
			Se \( X \) é \( \mathcal{G} \)-mensurável, então \( \mathbb{E}(X \mid \mathcal{G}) = X \) quase certamente.
			
			\item \textbf{Linearidade:}
			\[
			\mathbb{E}(a_1 X_1 + a_2 X_2 \mid \mathcal{G}) = a_1 \mathbb{E}(X_1 \mid \mathcal{G}) + a_2 \mathbb{E}(X_2 \mid \mathcal{G}), \quad \text{quase certamente.}
			\]
			
			\item \textbf{Positividade:}  
			Se \( X \geq 0 \), então \( \mathbb{E}(X \mid \mathcal{G}) \geq 0 \), quase certamente.
			
			\item \textbf{Monotonicidade (cMON):}  
			Se \( 0 \leq X_n \uparrow X \), então \( \mathbb{E}(X_n \mid \mathcal{G}) \uparrow \mathbb{E}(X \mid \mathcal{G}) \), quase certamente.
			
			\item \textbf{Convergência Dominada (cDOM):}  
			Se \( |X_n(\omega)| \leq V(\omega) \) para todo \( n \), \( \mathbb{E}[V] < \infty \), e \( X_n \to X \) quase certamente, então
			\[
			\mathbb{E}(X_n \mid \mathcal{G}) \to \mathbb{E}(X \mid \mathcal{G}), \quad \text{quase certamente.}
			\]
			
			\item \textbf{Desigualdade de Jensen (cJENSEN):}  
			Se \( \phi: \mathbb{R} \to \mathbb{R} \) é convexa e \( \mathbb{E}|\phi(X)| < \infty \), então
			\[
			\mathbb{E}[\phi(X) \mid \mathcal{G}] \geq \phi(\mathbb{E}[X \mid \mathcal{G}]), \quad \text{quase certamente.}
			\]
		\end{enumerate}
	\end{lemma}
		
				
							
				\end{frame}
	
	
		\begin{frame}{Esperança condicional: propriedades básicas}
		
		No que segue, tome $X \in \mathcal{L}^1(\Omega,\Sigma,\mathbb{P})$, e $\mathcal{F}$ e $\mathcal{G}$ sub-$\sigma$-álgebras de $\Sigma$.
		\begin{lemma}
			\begin{enumerate}
			\item[7] \textbf{Propriedade da Torre:}  
			Se \( \mathcal{H} \) é uma sub-\(\sigma\)-álgebra de \( \mathcal{G} \), então
			\[
			\mathbb{E}[\mathbb{E}(X \mid \mathcal{G}) \mid \mathcal{H}] = \mathbb{E}(X \mid \mathcal{H}), \quad \text{quase certamente}.
			\]
			
			\item[8] \textbf{'Extraindo o que é Conhecido':}  
			Se \( Z \) é \( \mathcal{G} \)-mensurável e limitada, então
			\[
			\mathbb{E}[ZX \mid \mathcal{G}] = Z \mathbb{E}[X \mid \mathcal{G}], \quad \text{quase certamente}.
			\]
				(resultado também vale se $X\in L^p$ e $Z \in L^q$, com $p\geq1$ e $\frac{1}{q} = 1-\frac{1}{p}$).
			
			\item[9] \textbf{Papel da Independência:}  
			Se \( \mathcal{H} \) é independente de \( \sigma(X, \mathcal{G}) \), então
			\[
			\mathbb{E}[X \mid \sigma(\mathcal{G}, \mathcal{H})] = \mathbb{E}(X \mid \mathcal{G}), \quad \text{quase certamente}.
			\]
			Em particular, se \( X \) é independente de \( \mathcal{H} \), então
			\[
			\mathbb{E}(X \mid \mathcal{H}) = \mathbb{E}(X), \quad \text{quase certamente}.
			\]
			\end{enumerate}
		\end{lemma}
		
		
		
	\end{frame}
	
		\begin{frame}{Probabilidade condicional}
\begin{itemize}
	\item Seja $(\Omega, \Sigma,\mathbb{P})$ um espaço de probabilidade, $A \in \Sigma$ um evento e $\mathcal{G}\subseteq \Sigma$ uma sub-$\sigma$-álgebra de $\Sigma$. Definimos a \textbf{probabilidade condicional} de $A$ dado $\mathcal{G}$ como:
	$$\mathbb{P}[A|\mathcal{G}] = \mathbb{E}[\mathbf{1}_{A}|\mathcal{G}]$$
\item Uma função $p: \Sigma \times \Omega \mapsto [0,1]$ é dita uma \textbf{probabilidade condicional regular} se:
\begin{enumerate}
	\item $\forall A \in \Sigma$, $\omega \mapsto p(A,\omega)$ é uma versão de $\mathbb{P}[A|\mathcal{G}]$.
	\item $\forall \omega \in \Omega$, $A \mapsto p(A, \omega)$ é uma lei de probabilidade sobre $(\Omega, \Sigma)$.
\end{enumerate}
\item Probabilidades condicionais regulares nem sempre existem, embora, para espaços bem-comportados como os estudados em Estatística, elas quase existam na maioria dos casos.
\end{itemize}
\end{frame}
\begin{frame}{Medida produto}
	\begin{itemize}
		\item Sejam $(\Omega_1,\Sigma_1,\mathbb{P}_1)$ e $(\Omega_2,\Sigma_2,\mathbb{P}_2)$ dois espaços de probabilidade.
		\item A $\sigma$-álgebra \textbf{produto}, denotada por $\Sigma_1 \otimes \Sigma_2 $, é a $\sigma$-álgebra em $\Omega_1 \times \Omega_2$ gerada pelos conjuntos da forma $B_1 \times B_2$, com $B_1,\Sigma_1$ e $B_2, \Sigma_2$.
		\item A medida produto, denotada por $\mathbb{P}_1\otimes \mathbb{P}_2$ é a medida caracterizada pelas probabilidades:
		$$\mathbb{P}_1\otimes \mathbb{P}_2[B_1 \times B_2] = \mathbb{P}_1[B_1] \mathbb{P}_1[B_2],\quad \forall B_1 \in \Sigma_1,B_2 \in \Sigma_2\,.$$
		(note o paralelismo com o conceito de independência; estamos definindo um novo espaço de probabilidade em que os experimentos 1 e 2 ocorrem de forma independente).
	
	\end{itemize} 
\end{frame}
\begin{frame}{Teorema de Fubini}
	\begin{itemize}
	\item Seja $f \in L^1(\Omega_1\times \Omega_2,\Sigma_1 \otimes \Sigma_2,  \mathbb{P}_1\otimes \mathbb{P}_2)$. Uma pergunta que podemos ter é se podemos calcular a esperança:
$$\int f d \mathbb{P}_1 \otimes \mathbb{P}_2\,,$$
de forma sequencial, isto é se:
\begin{equation*}
	\begin{aligned}
		\int f d \mathbb{P}_1 \otimes \mathbb{P}_2\ = \int \left(\int f(\omega_1,\omega_2)\mathbb{P}[d\omega_1]\right) \mathbb{P}[d\omega_2] = \\ \int \left(\int f(\omega_1,\omega_2)\mathbb{P}[d\omega_2]\right) \mathbb{P}[d\omega_1] 
	\end{aligned}
\end{equation*}
\item Resposta é \textbf{afirmativa}, e dada pelo \textbf{Teorema de Fubini}.
\begin{itemize}
	\item Teorema adicionalmente garante que podemos ``trocar'' as integrais.
	\item Teorema vale para expectativas e, mais genericamente, integrais de medidas $\sigma$-finitas.
\end{itemize}
\end{itemize}
\end{frame}
\end{document}
